% v2-acmsmall-sample.tex, dated March 6 2012
% This is a sample file for ACM small trim journals
%
% Compilation using 'acmsmall.cls' - version 1.3 (March 2012), Aptara Inc.
% (c) 2010 Association for Computing Machinery (ACM)
%
% Questions/Suggestions/Feedback should be addressed to => "acmtexsupport@aptaracorp.com".
% Users can also go through the FAQs available on the journal's submission webpage.
%
% Steps to compile: latex, bibtex, latex latex
%
% For tracking purposes => this is v1.3 - March 2012

\documentclass[prodmode,acmtecs]{acmsmall} % Aptara syntax

% Package to generate and customize Algorithm as per ACM style
\usepackage[linesnumbered, ruled]{algorithm2e}
\usepackage{amsfonts}
\usepackage{amsmath}
\usepackage{graphicx}
\graphicspath{ {data/} }

\SetFuncSty{textsc}
\SetKwInOut{Input}{input}\SetKwInOut{Output}{output}
\SetKwInOut{Require}{require}
\SetKwComment{Comment}{$\triangleright$~}{}

\renewcommand{\algorithmcfname}{ALGORITHM}
\SetAlFnt{\small}
\SetAlCapFnt{\small}
\SetAlCapNameFnt{\small}
\SetAlCapHSkip{0pt}
\IncMargin{-\parindent}
\newcommand{\todo}[1]{\texttt{\color{blue}TODO:#1}}
\newcommand{\fixme}[1]{\texttt{\color{red}FIXME:#1}}

\acmArticle{EE837}

% Document starts
\begin{document}

% Title portion
\title{CRIL: Compact Representation for Incremental Learning}
\author{Minsuk Choi
\affil{Electrical Engineering}
Seongmin Lee
\affil{School of Computing}
YoungKi Hong
\affil{School of Computing}}

\input{abstract}

\maketitle

\section{Introduction}
\label{sec:introduction}

Introduction

As the goal of the computer vision field is getting closer to artificial intelligence, more flexible strategies are needed to handle the large-slace and dynamic properties of real-world situations. As part of this, the visual object classification system should be able to incrementally accept and train new classes. We call this scenario as class-incremental learning~\cite{Rebuffi:2016aa}.

iCaRL is a state of the art strategy that allows learning in such a class-incremental way. It handles the crux by 1) sampling representative data with \textit{prioritized exemplar selection} algorithm and 2) classifying by a \textit{nearest-mean-of-exemplars} rule.

Despite the class-incremental learning has a limitation on a budget to store the exampler set, iCaRL has no algorithmic strategy to learn effective feature representation, therefore the capability of the exampler set to represent the class weaken as the the number of observed class increases.

In this paper, we propose CRIL, a compact representation on incremental learning. CRIL uses the VAE to train the feature extractor model so that data per class can be compactly represented on the feature space. This representation learning strategy will prevent the decline of the classification competence of reduced exampler set.

\todo{update here as we finalize the experiments}


우리는 MNIST 데이터셋으로 maximum exampler 를 줄여가며 icarl의 퍼포먼스가 얼마나 낮아지는지 확인하고 같은 환경헤서 CRIL을 실험하여 결과를 비교하였다. 또한 t-sne를 이용해서 CRIL로 learning 된 모델이 data를 feature 공간에 얼마나 compact하게 위치시키는지 visualization 해 보았다. The result shows ...

The main contribution of this paper is follows:
\begin{itemize}
\item We introduced CRIL which successfully models compact representation of individuals on the feature space using VAE technique.
\item By Comparing with iCarl with various size of exampler, we showed CRIL outperforms with the previous state-of-art technique.
\item \todo{t-sne로 시각적으로 확인}
\end{itemize}


\section{Background}
\label{sec:background}

\subsection{Class-incremental learning}
\label{sec:cil}

As the goal of the computer vision field is getting closer to artificial intelligence, more flexible strategies are needed to handle the large-slace and dynamic properties of real-world situations. As part of this, the visual object classification system should be able to incrementally accept and train new classes. We call this scenario as class-incremental learning.

\begin{figure}[h]
\includegraphics[width=80mm]{data/class-incremental_learning.png}
\centering
\caption{Class-incremental learning: an algorithm learns continuously from a sequential data stream in which new classes occur. At any time, the learner is able to perform multi-class classification for all classes observed so far. \label{fig:class-incremental_learning}}
\end{figure}

There are three formal properties of an algorithm to qualify as class-incremental:
\begin{enumerate}
\item it should be trainable from a stream of data in which examples of different classes occur at different times,
\item it should at any time provide a competitive multi-class classifier for the classes observed so far,
\item its computational requirements and memory footprint should remain bounded, or at least grow very slowly, with respect to the number of classes seen so far.
\end{enumerate}
The first and second properties directly express the essence of class-incremental learning. The third criterion concerns about practicality and prevents trivial algorithms; the algorithms such as storing all training examples and retraining an ordinary classifier whenever new data arives is meaningless.


\subsection{iCaRL: Incremental Classifier and Representation Learning}
\label{sec:icarl}

\begin{algorithm}[ht]
  \Input{$X^s, ..., X^t$ \Comment*[f]{training examples in per-class sets}}  
  \Input{$K$ \Comment*[f]{memory size}}  
  \Require{$\Theta$ \Comment*[f]{current model parameters}}
  \Require{$\mathcal{P} = \left(P_1, ..., P_{s-1} \right)$ \Comment*[f]{current exampler sets}}
  $\Theta \leftarrow \textsc{UpdateRepresentation}\left(X^s, ..., X^t;\mathcal{P};\Theta\right)$ \\
  $m \leftarrow K/t$ \Comment*[f]{number of exemplars per class} \\
  \For {$y = 1..s-1$}{
    $P_y \leftarrow \textsc{ReduceExemplarSet}\left(P_y,m\right)$ 
  }
  \For {$y = s..t$}{
    $P_y \leftarrow \textsc{ConstructExemplarSet}\left(X_y,m,\Theta\right)$ 
  }
  $\mathcal{P} \leftarrow \left(P_1, ..., P_t\right)$ \Comment*[f]{new exemplar sets} \\
\caption{ iCaRL \textsc{IncrementalTrain} \label{alg:icarl_learn}}
\end{algorithm}


iCaRL, an incremental classifier and representation learning, provides a practical strategy for simultaneous learning of classifiers and a feature representation in the class-incremental setting. Algorithm~\ref{alg:icarl_learn} shows the overall process of the training step of iCaRL.

There are three main components on its strategy to fulfill the criteria of class-incremental learning. These three components are:

\begin{itemize}
  \item classification by a \textit{nearest-mean-of-exemplars} rule,
  \item \textit{prioritized exemplar selection} based on herding,
  \item representation learning using \textit{knowledge distillation} and \textit{prototype rehearsal}.
\end{itemize}

iCaRL learns classifiers and feature represeantation simulaneously from on a data stream in class-incremental form, \textit{i.e.} sample sets $X^1, X^2, ...$, where all examples of a set $X^y = \left\{ x_1^y, ..., x_{n_y}^y \right\}$ are of class $y \in \mathbb{N}$.

\subsubsection{Classification}
\label{sec:icarl_classification}

iCaRL relies on \textit{exemplar images} sets $P_1, ..., P_t$ that it select dynamically out of the data stream for each observed class so far. iCaRL ensures that the toal number fo exemplar images never exceeds a fixed parameter $K$.

\begin{algorithm}[ht]
  \Input{$x$ \Comment*[f]{image to be classified}}  
  \Require{$\mathcal{P} = \left(P_1, ..., P_t \right)$ \Comment*[f]{class exemplar sets}}
  \Require{$\phi : \mathcal{X} \rightarrow \mathbb{R}^d$ \Comment*[f]{feature map}}
  \For {$y = 1, ..., t $}{
    $\mu_y \leftarrow \frac{1}{\left| P_y \right|} \sum\limits_{y \in P_y} \phi\left( p \right)$ \Comment*[f]{mean-of-exemplars}
  }
  $y^{*} \leftarrow \underset{y = 1, ..., t}{\textrm{argmin}} \left\| \phi(x) - \mu_y \right\|$ \Comment*[f]{nearest prototype} \\
  \Output{class label $y^{*}$}
  
\caption{ iCaRL \textsc{Classify} \label{alg:icarl_classify}}
\end{algorithm}

Algorithm~\ref{alg:icarl_classify} describes the mean-of-exempars classifier. To predict a label, $y^*$, for a new image, $x$, it computes a prototype vector for each class observed so far, $\mu_1, ..., \mu_t$, where $\mu_y = \frac{1}{\left| P_y \right|} \sum_{y \in P_y} \phi\left( p \right)$ is the average feature vector of all exemplars for a class $y$. The class label whose prototype vector is the most similar with $x$ is assigned to $y^*$

\subsubsection{Representation Learning}
\label{sec:icarl_learning}

Whenever iCaRL gets data, $X^s, ..., X^t$ for new classes, $s, ..., t$, it updates its features extraction routine and the exemplar set. Algorithm~\ref{alg:icarl_update} lists the step for incrementally improving the feature representation. It mainly aims to prevent or at least mitigate \textit{catastrophic fogetting}~\cite{McCloskey:1989aa}, which is a deterioration of classification accuracy. It augmented the traning set so that it consists not only of the new training examples but also of the stored exemplars. The lass function is augmented, too. It contains both the standard classification loss, which encourages improvements of the feature representation that allow classifying the newly observed classes, and the distillation loss, which ensures that the discriminative information learned previously is not lost during the new learning step.

\begin{algorithm}[ht]
  \Input{$X^s, ..., X^t$ \Comment*[f]{traning images of classes $s, ..., t$}}
  \Require{$\mathcal{P} = \left( P_1, ..., P_{s-1}\right)$ \Comment*[f]{exemplar sets}}
  \Require{$\Theta$ \Comment*[f]{ current model parameters}}

  $\mathcal{D} \leftarrow \bigcup\limits_{y=s..t}\left\{ (x,y) : x \in X^y \right\} \cup \bigcup\limits_{y=1..s-1}\left\{ (x,y) : x \in P^y \right\} $ \Comment*[f]{form combined traning set} \\
  \For{$y = 1, ..., s-1$}{ 
    $q_i^y \leftarrow g_y(x_i) \forall (x_i,\cdot) \in \mathcal{D}$ \Comment*[f]{store network outputs with pre-update parameters} \\
  } 
$l(\Theta) = - \sum\limits_{\left(x_i,y_i\right) \in \mathcal{D}} \left[ \sum\limits_{y=s}^t \delta_{y=y_i}\log g_y(x_i) + \delta_{y\neq y_i}\log \left(1-g_y(x_i)\right) + \sum\limits_{y=1}^{s-1} q_{i}^y\log g_y(x_i) + (1-q_i^y)\log \left(1-g_y(x_i)\right) \right]$ \Comment*[f]{run network traniing (\textit{e.g.} BackProp) with loss function that consists of \textit{classification} and \textit{distillation} terms.}

\caption{ iCaRL \textsc{UpdateRepresentation} \label{alg:icarl_update}}
\end{algorithm}

\subsubsection{Exampler Management}
\label{sec:icarl_exampler}

\begin{algorithm}[ht]
  \Input{image set $X= \{x_1, ..., x_n\}$ of class $y$}
  \Input{$m$ target number of exemplars}
  \Require{current feature function $\phi : \mathcal{X} \rightarrow \mathbb{R}^d$}

  $\mu \leftarrow \frac{1}{n}\sum\limits_{x\in X} \phi(x)$ \Comment*[f]{current class mean} \\
  \For{$k = 1, ..., m$}{
    $p_k \leftarrow \underset{x \in X}{\textrm{argmin}} \left\| \mu - \frac{1}{k}\left[ \phi(x) + \sum\limits_{j=1}^{k-1} \phi(p_j) \right] \right\|$
  }
  $P \leftarrow (p_1, ..., p_m)$

  \Output{exemplar set $P$}
\caption {iCaRL \textsc{ConstructExemplarSet} \label{alg:icarl_exampler_set}}
\end{algorithm}

Whenever iCaRL constructs exemplar sets for each classes, it treats all classes equally; iCaRL will use $m - K/t$ exemplars for each classes when it has been observed $t$ classes and the budget of the total number of the examplers is $K$.

Algorithm~\ref{alg:icarl_exampler_set} describes the examplar selection step. The top $m$ exemplars, whose average feature vector over all exemplars best approximates the average feature vector over all traning examples, are chosen. The reduction step is simple; on the sorted exemplars by aforementioned criterion, one discards the exemplars from the back until it reaches to desire size.



\section{CRIL: Compact Representation for Incremental Learning}
\label{sec:cril}

\subsection{Motivation}
\label{sec:motivation}
Motivation

\subsection{VAE}
\label{sec:vae}
VAE

\subsection{Algorithm} % OR Structure OR Architecture
\label{sec:algorithm}
Algorithm

\section{Experimantal Setup}
\label{sec:setup}

\subsection{Research Questions}
\label{sec:rq}

\subsubsection{RQ1: }
\label{sec:rq1}
RQ1

\subsubsection{RQ2: }
\label{sec:rq2}
RQ2

\subsubsection{RQ3: }
\label{sec:rq3}
RQ3

\subsection{Subject and Environment}
\label{sec:subject}
\todo{YOUNGKI PLEASE HELP}

Subject
Env
MNIST

\subsection{Configuration}
\label{sec:configuration}
\todo{YOUNGKI PLEASE HELP}

\todo{I don't know how to state the configuration on such like ML papers}

Config

\section{Results}
\label{sec:results}

\subsection{RQ1: }
\label{sec:results_rq1}
Results RQ1

\subsection{RQ2: }
\label{sec:results_rq2}
Results RQ2

\subsection{RQ3: }
\label{sec:results_rq3}
Results RQ3
\section{Conclusion}
\label{sec:conclusion}

We presents CRIL, compact representation for incremental learning, which is a brand new strategy for Class-Incremental Learning. With VAE, the network is trained to have the output features that follow the gaussian distribution with given mean and variance for each class. The network is expected to generate the output feature vectors which are more compact within each class and distinctive between classes.

The experiment was held to compare the effectiveness between current state-of-the-art Class-Incremental Learning strategy, iCaRL. The results shows that CRIL outperformed over 21\% in average accuracy to iCaRL for all sizes of exampler on MNIST dataset. Especillay for small size of exampler($K=100$), CRIL shows 75\% of test accuracy for 10 classes with the budget size($K$) 100 which was 23\% higher than iCaRL(52\%).

However, while both iCaRL and CRIL had poor performances(lower than 21\% of accuracy on average), iCaRL outperformed about 10\% in average accuracy to CRIL. We disscuss that adding the convolustional layer on CRIL may increase the performance.

\todo{t-sne}

\todo{YOUNGKI PLEASE HELP}


\bibliographystyle{ACM-Reference-Format-Journals}
\bibliography{papers}

\end{document}
% End of v2-acmsmall-sample.tex (March 2012) - Gerry Murray, ACM


